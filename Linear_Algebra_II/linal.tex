
\documentclass[15pt,a4paper]{book}

\usepackage{amsmath, amsthm, amssymb} 
\usepackage{graphicx} % For including graphics
\usepackage{hyperref} % For clickable links
\usepackage{bookmark} % Better control over bookmarks
\usepackage{geometry} % Customize page layout
\usepackage{xcolor} % Colors for text and graphics
\usepackage{enumitem} % Customizable lists
\usepackage{fancyhdr} % Header and footer
\usepackage{titlesec} % Custom section/chapter titles
\usepackage[toc,page]{appendix} % For the appendix
\usepackage{longtable} % For tables spanning multiple pages
\usepackage{mathrsfs} % For script fonts in math mode
\usepackage{tocloft} % Custom table of contents
\usepackage{datetime2} % For dates
\usepackage{caption} % For better control over captions
\usepackage{float} % Fine control over figure/table placement
\usepackage{imakeidx} % For index

% Custom Theorem Styles
\newtheorem{theorem}{Theorem}[chapter]
\newtheorem{lemma}[theorem]{Lemma}
\newtheorem{proposition}[theorem]{Proposition}
\newtheorem{corollary}[theorem]{Corollary}
\theoremstyle{definition}
\newtheorem{definition}[theorem]{Definition}
\newtheorem{example}[theorem]{Example}
\newtheorem{remark}[theorem]{Remark}

\renewcommand{\cftchapfont}{\normalfont} % Remove bold for chapter names
\renewcommand{\cftchappagefont}{\normalfont} % Remove bold for chapter page numbers
\renewcommand{\qedsymbol}{$\blacksquare$}
\newcommand{\eax}[1]{\emph{#1}\index{#1}} % Macro for emphasis and index
\newcommand{\abs}[1]{\left| #1 \right|} % Absolute value


% Custom Notation List Environment
\newlist{notationlist}{description}{1}
\setlist[notationlist]{font=\bfseries,labelsep=1em}

% Geometry Settings
\geometry{
    top=2.5cm,
    bottom=2.5cm,
    left=2.5cm,
    right=2.5cm,
}

% Hyperref Colors
\hypersetup{
    colorlinks=true,
    linkcolor=black,
    urlcolor=cyan,
    citecolor=red
}

\renewcommand{\chaptermark}[1]{\markboth{#1}{}}
\renewcommand{\sectionmark}[1]{\markright{#1}}

% Custom Headers
\pagestyle{fancy}
\fancyhf{}
\fancyhead[L]{\leftmark} % Chapter name on top left
\fancyhead[R]{\rightmark}  % section name on top right
\fancyfoot[C]{\thepage}

\renewcommand{\headrulewidth}{0pt}
\renewcommand{\footrulewidth}{0pt}

% Making index
\makeindex[intoc]

% Title Formatting
\titleformat{\chapter}[display]
  {\normalfont\Large\bfseries \centering}
  {\chaptername\ \thechapter}{20pt}{\Huge \centering}

\titlespacing*{\chapter}{0pt}{20pt}{100pt}

\begin{document}

\pagestyle{empty}

\begin{titlepage}
    \begin{center}
    \vspace*{\fill}
    % Title in all caps
    {\Huge \textbf{\MakeUppercase{Linear Algebra II}}\par}

    \vspace{0.5cm} % Adjust vertical spacing between title and subtitle
    % Subtitle in normal text, slightly enlarged
    {\Large Anita Naolekar, notes by Ramdas Singh\par}

    \vspace{0.5cm} % Additional spacing before the author
    % Author information
    {\large Second Semester\par}
    \vspace*{\fill}
    \end{center}
\end{titlepage}

\clearpage

\pagenumbering{roman}

\chapter*{List of Symbols}
\begin{notationlist}
    \item
\end{notationlist}

\newpage
\setcounter{tocdepth}{2}
\tableofcontents

\newpage
\pagenumbering{arabic}
\pagestyle{fancy}


%%-------------------------------------------------------------------------------------------------


\chapter{PERMUTATION GROUPS}


\textit{January 3rd.}

Let $S_{n}$ denote the set of all bijections (permutations) on the set $\{1,2,\ldots,n\}$. If $\sigma, \tau \in S_{n}$, let us define $\sigma \tau$ to be the bijection defined as
\begin{equation}
    (\sigma \tau)(i) = \sigma(\tau(i)) \forall 1 \leq i \leq n.
\end{equation}

This gives us a binary operation on $S_{n}$ which is associative, and $S_{n}$ will then contain the identity permutation 1 such that $\sigma 1 = 1 \sigma = \sigma$ for all $\sigma \in S_{n}$. For every such $\sigma$, we can also find a $\sigma^{-1} \in S_{n}$ such that $\sigma \sigma^{-1} = \sigma^{-1}\sigma = 1$. The set $S_{n}$ equipped with this binary operation, thus, forms a group. In this case, we call $S_{n}$ as the \eax{symmetric group} of degree $n$. We now define a cycle in regards to permutations.

\begin{definition}
    A \eax{cycle} is a a string of positive integers, say $(i_{1},i_{2},\ldots,i_{k})$, which represents the permutation $\sigma \in S_{n}$ (with $k\leq n$) such that $\sigma(i_{j})=i_{j+1}$ for all $1 \leq j \leq k-1$, and $\sigma(i_{k}) = i_{1}$, and fixes all other integers. 
\end{definition}

We also note that $S_{3}$ is the smallest Abelian group possible, upto isomorphism. $S_{3}$ is one of the only two groups of order 6, and can be written as
\begin{equation}
    S_{3} = \{1, \sigma=(1,2,3), \sigma^{2}=(1,3,2), \tau=(1,2),\sigma\tau=(1,3),\tau\sigma=(2,3)\}.
\end{equation}
Some other observations arise. We find that $\sigma^{3} = \tau^{2} = 1$, and that $\tau\sigma = \sigma^{2}\tau$. We notice another fact via this $\sigma$;

\begin{remark}
    A \eax{k-cycle} $\sigma=(i_{1},i_{2},\ldots,i_{k})$ is of order $k$, that is, $\sigma^{k}=1$.
\end{remark}

\begin{definition}
    Two cycles in $S_{n}$ are called disjoint if they have no intger in common.
\end{definition}
We note that if $\sigma$ and $\tau$ are two disjoint cycles in $S_{n}$ then $\sigma$ and $\tau$ commute, that is, $\sigma \tau = \tau \sigma$.

\begin{proposition}
    Every $\sigma$ in $S_{n}$ can be written uniquely as a product of disjoint cycles.
\end{proposition}
Every cycle can be written as a product of 2-cycles. 2-cycles are called \eax{transpositions}. This can easily be seen as
\begin{equation}
    (a_{1},a_{2},\ldots,a_{n})=(a_{1},a_{n})(a_{1},a_{n-1})\cdots(a_{1},a_{3})(a_{1},a_{2}).
\end{equation}

\section{Even and Odd Permutations}
Let $x_{1},x_{2},\ldots,x_{n}$ be indeterminates, and let
\begin{equation}
    \Delta = \prod_{1 \leq i < j \leq n} (x_{i}-x_{j}).
\end{equation}
Let $\sigma \in S_{n}$, and define 
\begin{equation}
    \sigma(\Delta) = \prod_{1 \leq i < j \leq n} (x_{\sigma(i)}-x_{\sigma(j)}).
\end{equation}
We find that $\sigma(\Delta) = \pm \Delta$. Based on this, we classify permutations as odd or even.

\begin{definition}
    A permutation $\sigma$ is said to be an \eax{even permutation} if $\sigma(\Delta) = \Delta$, and is said to be an \eax{odd permutation} if $\sigma(\Delta) = -\Delta$. The sign of a permutation $\sigma$, denoted by $\epsilon(\sigma)$, is $+1$ if $\sigma$ is even, and is $-1$ if $\sigma$ is odd. So, $\sigma(\Delta) = \epsilon(\sigma)\Delta$.
\end{definition}

\begin{proposition}
    The map $\epsilon: S_{n} \to \{-1,+1\}$, where $\epsilon(\sigma)$ is the sign of $\sigma$, is a homomorphism, that is, $\epsilon(\sigma \tau) = \epsilon(\sigma)\epsilon(\tau)$ for all $\sigma, \tau \in S_{n}$.
\end{proposition}
\begin{proof}
    Start with $\tau(\Delta)$;
    \begin{equation}
        \tau(\Delta) = \prod_{1 \leq i < j \leq n} (x_{\tau(i)}-x_{\tau(j)}).
    \end{equation}
    Let there be $k$ factors of this polynomial where $\tau(i)>\tau(j)$ with $i<j$. We find that $\tau(\Delta) = (-1)^{k}\Delta$, and so, $\epsilon(\tau) = (-1)^{k}$. Now, $\sigma \tau(\Delta)$ has exactly $k$ factors of the form $x_{\sigma(j)}-x_{\sigma(i)}$, with $j > i$. Bringing out a factor $(-1)^{k}$, we find that $\sigma \tau (\Delta)$ has all factors of the form $x_{\sigma(i)}-x_{\sigma(j)}$, with $j > i$. Thus,
    \begin{equation}
        \epsilon(\sigma \tau)\Delta = \sigma \tau (\Delta) = (-1)^{k} \prod_{1 \leq i < j \leq n} (x_{\sigma(i)}-x_{\sigma(j)}) = (-1)^{k} \sigma(\Delta) = (-1)^{k} \epsilon(\sigma) \Delta = \epsilon(\tau) \epsilon(\sigma) \Delta.
    \end{equation}
    Cancelling out the $\Delta$, we find $\epsilon(\sigma \tau) = \epsilon(\sigma) \epsilon(\tau)$.
\end{proof}
$\epsilon$ is a homomorphism to an Abelian group, so $\epsilon(\sigma \tau) = \epsilon(\sigma) \epsilon(\tau) = \epsilon(\tau) \epsilon(\sigma)$.

\begin{proposition}
    If $\lambda = (i,j)$ is a transposition, then $\epsilon(\lambda) = -1$.
\end{proposition}
\begin{proof}
    If $\lambda = (1,2) \in S_{n}$, it is easy to show that
    \begin{equation}
        \lambda(\Delta) = (x_{1}-x_{2}) \cdots (x_{1}-x_{n}) (x_{2}-x_{3}) \cdots (x_{2}-x_{n}) \cdots = (-1)(\Delta).
    \end{equation}
    Now, if $\sigma = (i,j)$, with $(i,j) \neq (1,2)$, then $(i,j) = \lambda (1,2) \lambda$ where $\lambda$ interchanges $1$ and $i$, and interchanges $2$ and $j$. Using that fact that $\epsilon$ is a homomorphism, $\epsilon(\sigma) = -1$.
\end{proof}

A cycle $\sigma$ of length $k$ is an even permutation if and only if $k$ is odd. This is because it can be decomposed into $k-1$ transpositions, and we would then have $\epsilon(\sigma) = (-1)^{k-1} = 1$ (using the fact that $\epsilon$ is a homomorphism). Some more corollaries of the previous proposition include the fact that $\epsilon$ is a surjective map, and that $\epsilon(\sigma^{-1}) = \epsilon(\sigma)$.

If, for $\sigma \in S_{n}$, $\sigma$ can be decomposed as $\sigma_{1}\sigma_{2} \cdots \sigma_{k}$, where $\sigma_{i}$ is a $m_{i}$-cycle, then $\epsilon(\sigma_{i}) = (-1)^{m_{i}-1}$, and $\epsilon(\sigma) = (-1)^{(\sum m_{i}) - k}$.

\begin{proposition}
    $\sigma$ is an odd permutation if and only if the number of cycles of even length in its cycle decomposition is odd.
\end{proposition}

\begin{definition}
    If $A = (a_{ij})$ is a square matrix of order $n$, then the \eax{determinant} of $A$ is defined as
    \begin{equation}
        \det{A} = \sum_{\sigma \in S_{n}} \epsilon(\sigma) a_{1 \sigma(1)} a_{2 \sigma(2)} \cdots a_{n \sigma(n)}.
    \end{equation}
\end{definition}
Using this definition of the determinant of a square matrix, one may derive the usual determinant properties with ease.

\begin{appendices}

\titleformat{\chapter}[display]
  {\normalfont\Large\bfseries}
  {\chaptername\ \thechapter}{20pt}{\Huge}

\titlespacing*{\chapter}{0pt}{20pt}{40pt}

\chapter{Appendix}
Extra content goes here.

\printindex

\end{appendices}

\end{document}
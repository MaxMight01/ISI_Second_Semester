
\documentclass[15pt,a4paper]{book}

\usepackage{amsmath, amsthm, amssymb} 
\usepackage{graphicx} % For including graphics
\usepackage{hyperref} % For clickable links
\usepackage{bookmark} % Better control over bookmarks
\usepackage{geometry} % Customize page layout
\usepackage{xcolor} % Colors for text and graphics
\usepackage{enumitem} % Customizable lists
\usepackage{fancyhdr} % Header and footer
\usepackage{titlesec} % Custom section/chapter titles
\usepackage[toc,page]{appendix} % For the appendix
\usepackage{longtable} % For tables spanning multiple pages
\usepackage{mathrsfs} % For script fonts in math mode
\usepackage{tocloft} % Custom table of contents
\usepackage{datetime2} % For dates
\usepackage{caption} % For better control over captions
\usepackage{float} % Fine control over figure/table placement
\usepackage{imakeidx} % For index

% Custom Theorem Styles
\newtheorem{theorem}{Theorem}[chapter]
\newtheorem{lemma}[theorem]{Lemma}
\newtheorem{proposition}[theorem]{Proposition}
\newtheorem{corollary}[theorem]{Corollary}
\theoremstyle{definition}
\newtheorem{definition}[theorem]{Definition}
\newtheorem{example}[theorem]{Example}
\newtheorem{remark}[theorem]{Remark}

\renewcommand{\cftchapfont}{\normalfont} % Remove bold for chapter names
\renewcommand{\cftchappagefont}{\normalfont} % Remove bold for chapter page numbers
\renewcommand{\qedsymbol}{$\blacksquare$}
\newcommand{\eax}[1]{\emph{#1}\index{#1}} % Macro for emphasis and index
\newcommand{\abs}[1]{\left| #1 \right|} % Absolute value
\newcommand{\N}{\mathbb{N}} % Natural Numbers
\newcommand{\R}{\mathbb{R}} % Real numbers
\newcommand{\Z}{\mathbb{Z}} % Integers
\newcommand{\cP}{\mathcal{P}}
\newcommand{\cR}{\mathcal{R}}
\newcommand{\cB}{\mathcal{B}}
\newcommand{\cC}{\mathcal{C}}
\newcommand{\norm}[1]{\left\lVert#1\right\rVert}


% Custom Notation List Environment
\newlist{notationlist}{description}{1}
\setlist[notationlist]{font=\bfseries,labelsep=1em}

% Geometry Settings
\geometry{
    top=2.5cm,
    bottom=2.5cm,
    left=2.5cm,
    right=2.5cm,
}

% Hyperref Colors
\hypersetup{
    colorlinks=true,
    linkcolor=black,
    urlcolor=cyan,
    citecolor=red
}

\renewcommand{\chaptermark}[1]{\markboth{#1}{}}
\renewcommand{\sectionmark}[1]{\markright{#1}}

% Custom Headers
\pagestyle{fancy}
\fancyhf{}
\fancyhead[L]{\leftmark} % Chapter name on top left
\fancyhead[R]{\rightmark}  % section name on top right
\fancyfoot[C]{\thepage}

\renewcommand{\headrulewidth}{0pt}
\renewcommand{\footrulewidth}{0pt}

% Making index
\makeindex[intoc]

% Title Formatting
\titleformat{\chapter}[display]
  {\normalfont\Large\bfseries \centering}
  {\chaptername\ \thechapter}{20pt}{\Huge \centering}

\titlespacing*{\chapter}{0pt}{20pt}{100pt}

\begin{document}

\pagestyle{empty}

\begin{titlepage}
    \begin{center}
    \vspace*{\fill}
    % Title in all caps
    {\Huge \textbf{\MakeUppercase{Real Analysis II}}\par}

    \vspace{0.5cm} % Adjust vertical spacing between title and subtitle
    % Subtitle in normal text, slightly enlarged
    {\Large Jaydeb Sarkar, notes by Ramdas Singh\par}

    \vspace{0.5cm} % Additional spacing before the author
    % Author information
    {\large Second Semester\par}
    \vspace*{\fill}
    \end{center}
\end{titlepage}

\clearpage

\pagenumbering{roman}

\chapter*{List of Symbols}
\begin{notationlist}
    \item $[a,b]$, the set of all real numbers $x$ such that $a \leq x \leq b$.
    \item $\mathbb{N} = \{1,2,\ldots\}$, the set of all natural numbers.
    \item $\mathbb{Z}_{+}$, defined as $\mathbb{N} \cup \{0\}$.
    \item $\mathcal{B}[a,b]$, the set of all boundary functions defined as $\{f:[a,b] \to \mathbb{R}\}$. It is a vector space (also an algebra) over $\mathbb{R}$.
    \item $\mathcal{P}[a,b]$, the set of all partitions of the set $[a,b]$.
    \item $I_{j}$, the $j^{\text{th}}$ subinterval of $[a,b]$, controlled by a partition set.
    \item $L(f,P)$, the lower Riemann sum for a function $f$ and partition $P$.
    \item $U(f,P)$, the upper Riemann sum for a function $f$ and partition $P$.
    \item $\int_{\underline{a}}^{b} f$, the lower Riemann integration for a function $f$.
    \item $\int_{a}^{\overline{b}} f$, the upper Riemann integration for a function $f$.
    \item $\mathcal{R}[a,b]$, the set of all Riemann integrable functions over the set $[a,b]$.
    \item $\cC[a,b]$, the set of all continuous functions over the set $[a,b]$
    \item $\{a_{n}\}_{n \geq 1}$, a sequence of real numbers.
\end{notationlist}

\newpage
\setcounter{tocdepth}{2}
\tableofcontents

\newpage
\pagenumbering{arabic}
\pagestyle{fancy}


%%-------------------------------------------------------------------------------------------------

\chapter{THE RIEMANN INTEGRAL}

\section{On The Path of Definitions}

\textit{January 6th.}

\begin{definition}
    A \eax{partition} of $[a,b]$ are all the points $a=x_{0}<x_{1}<\ldots<x_{n} = b$. These points within are termed \eax{nodes}, and there are $n-1$ of them. The set $I_{j}$, defined by $[x_{j-1},x_{j}]$ denotes the $j^{\text{th}}$ subinterval.
\end{definition}

\begin{definition}
    If $I = (a,b), [a,b], (a,b], [b,a)$, then the \eax{length of the interval} $I$ is denoted by $b-a$.
\end{definition}

Denote by $\mathcal{P}[a,b]$, the set of all partition sets of $[a,b]$. For $P \in \mathcal{P}[a,b]$, with $n-1$ nodes, the length of $[a,b]$ will be $\abs{[a,b]} = \sum_{j=1}^{n} I_{j}$. We also note that for all $P, \tilde{P} \in \mathcal{P}[a,b]$, $P \cup \tilde{P} \in \mathcal{P}[a,b]$. Note that here we consider $n$ to be finite.

\begin{example}
    The set $\{\frac{1}{n}\}_{n \geq 1} \cup \{0\}$ does not belong to the set of all partitions of the unit interval, $\mathcal{P}[0,1]$.
\end{example}

Let $f \in \mathcal{B}[a,b]$, and $P \in \mathcal{P}[a,b]$. Suppose $P$ has the nodes $a = x_{0} < x_{1} < \ldots < x_{n} = b$. For all $j = 1, \ldots n$, define $m_{j} = \inf_{x \in I_{j}} f(x)$ and $M_{j} = \sup_{x \in I_{j}} f(x)$. Finally, denote by $m$ the value of $\inf_{x \in [a,b]} f(x)$ and $M$ to be $\sup_{x \in [a,b]} f(x)$. These are all real values.

Note that for all valid $j$, $m \leq m_{j} \leq M_{j} \leq M$ always holds. This must mean that
\begin{align}
    m \abs{I_{j}} &\leq m_{j} \abs{I_{j}} \leq M_{j} \abs{I_{j}} \leq M \abs{I_{j}} \notag \\
    m(b-a) &\leq \sum_{j=1}^{n} m_{j} \abs{I_{j}} \leq \sum_{j=1}^{n} M_{j} \abs{I_{j}} \leq M(b-a).
\end{align}

\begin{definition}
    Let $f \in \mathcal{B}[a,b]$. For $P$ $(a = x_{0}, x_{1}, \ldots, x_{n} = b)$ $\in \mathcal{P}[a,b]$, the \eax{lower Riemann sum} and the \eax{upper Riemann sum} are defined as
    \begin{equation}
        L(f,P) = \sum_{j=1}^{n} m_{j} \abs{I_{j}} \text{ and } U(f,P) = \sum_{j=1}^{n} M_{j} \abs{I_{j}},
    \end{equation}
    respectively. Thus, $m(b-a) \leq L(f,P) \leq U(f,P) \leq M(b-a)$ $\forall$ $P \in \mathcal{P}[a,b]$.
\end{definition}

\begin{remark}
    Clearly, $L(f,P), U(f,P) \in \mathbb{R}$ for all paritions $P \in \mathcal{P}[a,b]$ and all boundary functions $f \in \mathcal{B}[a,b]$. In fact, $L(f,P), U(f,P) \in [m(b-a),M(b-a)]$.
\end{remark}

\begin{definition}
    For $f \in \mathcal{B}[a,b]$, the \eax{lower Riemann integration} is defined as
    \begin{equation}
        \int_{\underline{a}}^{b} f = \sup\{L(f,P) | P \in \mathcal{P}[a,b]\}.
    \end{equation}
    Subsequently, the \eax{upper Riemann integration} is defined as
    \begin{equation}
        \int_{a}^{\overline{b}} f = \inf\{U(f,P) | P \in \mathcal{P}[a,b]\}.
    \end{equation}
\end{definition}
\begin{remark}
    Note that both $\int_{\underline{a}}^{b} f$ and $\int_{a}^{\overline{b}} f$ belong to the set $[m(b-a),M(b-a)]$.
\end{remark}

\begin{definition}
    A function $f \in \mathcal{B}[a,b]$ is \eax{Riemann integrable} if the lower and the upper Riemann integration are equal, that is, $\int_{\underline{a}}^{b} f = \int_{a}^{\overline{b}} f$. We denote this value by $\int_{a}^{b} f$, and call it the integration of $f$ over $[a,b]$. We then say that $f \in \mathcal{R}[a,b]$.
\end{definition}
\textit{January 8th.}
\begin{example}
    Note that $\cR [a,b] \subseteq \cB [a,b]$. In fact, it is a proper subset; for there exists the \eax{Dirichlet function} $f:[0,1] \to \R$ defined by
    \begin{equation}
        f(x) = \begin{cases}
            1 &\text{ if } x \in \mathbb{Q} \cap [0,1],\\
            0 &\text{ if otherwise.}
        \end{cases}
    \end{equation}
    Clearly, $f$ is a boundary function but not a continuous one. Now pick a partition $P$ with $x_{0}=0<x_{1}<\ldots<x_{n}=1$. Now, $m_{j}=0 \; \forall \; \implies L(f,P) = 0 \; \forall \; P \implies \int_{\underline{0}}^{1} f = 0$. If we consider that $M_{j}$ is always 1, we get $\int_{0}^{\overline{1}} f = 1$. $f$ does not belong to the set of Riemann integrable functions.
\end{example}
\begin{example}
    We show that $\cR [a,b]$ is not empty. Simply pick $f:[a,b] \to \R$ defined by $f(x) = c$ for all valid $x$. For every partition of this interval, we $m_{j} = M_{j} = c$ for all $j$. Finally, after computing the lower Riemann and upper Riemann sums, we get $\int_{\underline{a}}^{b} f = \int_{a}^{\overline{b}} f = c(b-a)$.
\end{example}
\begin{example}
    There exists a function $f \in \cB [a,b]$ such that $\abs{f} \in \cR [a,b]$ but $f \notin \cR [a,b]$. Indeed, simply pick a modification of the Dirichlet function defined as
    \begin{equation}
        f(x) = \begin{cases}
            -1 &\text{ if } x \in \mathbb{Q} \cap [a,b],\\
            1 &\text{ if otherwise.}
        \end{cases}
    \end{equation}
\end{example}

\begin{definition}
    Let $P, \tilde{P} \in \cP [a,b]$. We say $\tilde{P} \supset P$, or $\tilde{P}$ is a \eax{refinement} of $P$ if the nodes of $P$ are a subset of the nodes of $\tilde{P}$.
\end{definition}
\begin{example}
    For all $P,\tilde{P} \in \cP [a,b]$, we have $P \cup \tilde{P} \supset P, \tilde{P}$.
\end{example}
\begin{proposition}
    Let $f \in \cB [a,b]$, and $P, \tilde{P} \in cP [a,b]$. Suppose $\tilde{P} \supset P$. Then
    \begin{equation}
        L(f,P) \leq L(f,\tilde{P}) \leq U(f,\tilde{P}) \leq U(f,P).
    \end{equation}
\end{proposition}
\begin{proof}
    Note that it is sufficient to prove the first inequality; the second one is already true and the third one is analagous to the first one. Set $\tilde{P} = P \cup \{\tilde{x}\}$ with $\tilde{x} \notin P$, and set $P$ as $a = x_{0} < x_{1} < \ldots < x_{n} = b$. As $\tilde{x}$ is not part of $P$, there must exist some $j \in \{1, \ldots, n\}$ such that $\tilde{x} \in (x_{j-1},x_{j})$.

    For this $j$, let $\tilde{m}_{j-1} = \inf_{[x_{j-1},\tilde{x}]} f$ and let $\tilde{m}_{j} = \inf_{[\tilde{x},x_{j}]} f$. Therefore, we shall have
    \begin{align}
        L(f,\tilde{P}) - L(f,P) &= \tilde{m}_{j-1}(\tilde{x}-x_{j-1}) + \tilde{m}_{j}(x_{j}-\tilde{x}) - m_{j}(x_{j}-x_{j-1}) \notag \\
        &= \tilde{m}_{j-1}(\tilde{x}-x_{j-1}) + \tilde{m}_{j}(x_{j}-\tilde{x}) - m_{j}(x_{j}-\tilde{x}) - m_{j}(\tilde{x}-x_{j-1}) \notag \\
        &= (\tilde{m}_{j}-m_{j})(x_{j}-\tilde{x}) + (\tilde{m}_{j-1} - m_{j})(\tilde{x}-x_{j-1}) \geq 0 \notag \\
        \implies L(f,\tilde{P}) - L(f,P) &\geq 0.
    \end{align}
    Induction may now be applied to make any refinement $\tilde{P}$ of $P$. A similar logic applies to the upper Riemann sums.
\end{proof}
\begin{corollary}
    Let $f \in \cB [a,b]$. Then, for all $P,Q \in \cP [a,b]$, $L(f,P) \leq U(f,Q)$.
\end{corollary}
\begin{proof}
    Choose $R = P \cup Q$ to be a refinement of both $P$ and $Q$. Applying the previous proposition, we simply get $L(f,P) \leq L(f,R) \leq U(f,R) \leq U(f,Q)$.
\end{proof}
\begin{corollary}
    For all $f \in \cB [a,b]$, $\int_{\underline{a}}^{b} f \leq \int_{a}^{\overline{b}} f$ is always true.
\end{corollary}
\begin{proof}
    The lower Riemann intergral is the supremum of all the lower Riemann sums, so it must be the lowest upper bound, and, thus, has to be lesser than the upper Riemann sums. Similarly, the upper Riemann integral is greater than the lower Riemann sums. Consequently, we get the desired inequality.
\end{proof}
\begin{theorem}
    Let $f \in \cB [a,b]$. Then, for $f$ to be Riemann intergrable, the only neccessary and sufficient condition is $\int_{\underline{a}}^{b} f \geq \int_{a}^{\overline{b}} f$.
\end{theorem}
\begin{proof}
    If the condition is satisfied, then we must conclude that the Riemann integrals have to be equal. The converse follows the opposite argument.
\end{proof}
\begin{theorem}
    Let $f \in \cB [a,b]$. Then $f \in \cR [a,b]$ if and only if for every $\varepsilon > 0$, there exists a $P \in \cP [a,b]$ such that $U(f,P) - L(f,P) < \varepsilon$.
\end{theorem}
\begin{proof}
    We first prove the converse; assume that for every $\varepsilon > 0$, there exists a $P$ satisfying $U(f,P) - L(f,P) < \varepsilon$. Now,
    \begin{align}
        L(f,P) \leq \underline{\int} f \leq \overline{\int} f \leq U(f,P) < \varepsilon + L(f,P) \leq \varepsilon + \underline{\int} f \notag \\
        \implies \overline{\int} f - \underline{\int} f < \varepsilon \; \forall \; \varepsilon < 0\\
        \implies \overline{\int} f = \underline{\int} f.
    \end{align}
    To show that every Riemann integrable function satisfies this property, let $f \in \cR [a,b]$ and $\varepsilon > 0$. The Riemann integrals are a supremum and an infimum, so there must exist a $P_{1} \in \cP [a,b]$ such that $L(f,P_{1}) > \int f - \frac{\varepsilon}{2}$ and a $P_{2} \in \cP [a,b]$ such that $U(f,P_{1}) < \int f + \frac{\varepsilon}{2}$. Now choose $P$ to be $P_{1} \cup P_{2}$, a refinement of both $P_{1}$ and $P_{2}$. Therefore,
    \begin{align}
        U(f,P) \leq U(f,P_{2}) < \int f + \frac{\varepsilon}{2} < (L(f,P_{1}) + \frac{\varepsilon}{2}) + \frac{\varepsilon}{2} \leq L(f,P) + \varepsilon \notag \\
        \implies U(f,P) - L(f,P) < \varepsilon.
    \end{align}
\end{proof}

\begin{definition}
    Let $P$ be a partition with $a=x_{0} < x_{1} < \ldots < x_{n} = b$. We define the \eax{norm} of $P$, or the \eax{mesh} of $P$, as $\norm{P} = \max_{j}\{x_{j}-x_{j-1}\}$.
\end{definition}

\begin{theorem}[\eax{Darboux's theorem}]
    Let $f \in \cB [a,b]$. Then $f \in \cR [a,b]$ if and only if for every $\varepsilon > 0$, there exists a $\delta > 0$ such that $U(f,P)-L(f,P) < \varepsilon$ for all $P \in \cP [a,b]$ with $\norm{P} < \delta$. 
\end{theorem}
\begin{remark}
    To prove this, we define $\eta: \cP [a,b] \to \R_{\geq 0}$ by $\eta(P) = U(f,P) - L(f,P)$ for all $P \in \cP [a,b]$.
\end{remark}
\begin{proof}
    The proof of the converse is trivial and follows the same reasoning as before. To show that every Riemann integrable function satisfies this property, let $f \in \cR [a,b]$ and $\varepsilon > 0$. There exists a $\tilde{P} \in \cP [a,b]$ such that $U(f,\tilde{P}) - L(f,\tilde{P}) < \frac{\varepsilon}{2}$. Denote the number of nodes in $\tilde{P}$ by $p$, and set $\delta = \frac{\varepsilon}{8pM}$, where $M$ is the supremum of $f$ over $[a,b]$. Pick $P \in \cP [a,b]$ and assume that $\norm{P} < \delta$. Now set $\hat{P} = P \cup \tilde{P}$; $\hat{P}$ has at most $p$ points that are not in $P$.

    For now, assume that $p = 1$. Then, $\tilde{P} = P \cup \{\tilde{x}\}$ with $\tilde{x} \notin P$. Thus, with variables defined as before,
    \begin{equation}
        L(f,\hat{P}) - L(f,P) = (\tilde{m}_{j}-m_{j})(x_{j}-\tilde{x}) + (\tilde{m}_{j-1} - m_{j})(\tilde{x}-x_{j-1}).
    \end{equation}
    Notice that $(\tilde{m}_{j}-m_{j}), (\tilde{m}_{j-1} - m_{j}) < 2M$ and $(x_{j}-\tilde{x}), (\tilde{x}-x_{j-1}) < \delta$. In fact, in general, for an arbitrary $p$, we have
    \begin{equation}
        L(f,\hat{P}) - L(f,P) < 4pM\delta = \frac{\varepsilon}{2}.
    \end{equation}
    A similar story unfolds for the upper sums,
    \begin{equation}
        U(f,P) - U(f, \hat{P}) < \frac{\varepsilon}{2}.
    \end{equation}
    Together, the equations combine to form
    \begin{align}
        U(f,P) - L(f,P) < \varepsilon + U(f,\hat{P})-L(f,\hat{P}) < \varepsilon + U(f,\tilde{P})-L(f,\tilde{P}) < 2\varepsilon.
    \end{align}
\end{proof}
\textit{January 13th.}\\
We now wish to answer two questions; which elements reside in the set $\cR[a,b]$, and the value of the Riemann integral $\int_{a}^{b} f$ for some $f \in \cR[a,b]$.

Let us first wonder whether $\cC[a,b]$, the set of continuous functions, is a subset of $\cR[a,b]$. In fact, this is true.

\begin{theorem}
    $\cC[a,b] \subseteq \cR[a,b]$.
\end{theorem}
\begin{proof}
    Fix $f \in \cC[a,b]$; thus, $f:[a,b] \to \R$ is also uniformly continuous. For all $\varepsilon > 0$, there eixsts $\delta > 0$ such that
    \begin{equation}
        \abs{f(x)-f(y)} < \frac{\varepsilon}{b-a} \text{ for all } \abs{x-y} < \delta.
    \end{equation}
    Pick $P \in \cP[a,b]$ such that $\norm{P} < \delta$, and fix such a $P : a = x_{0} < x_{1} < \ldots < x_{n} = b$. Thus,
    \begin{align}
        U(f,P) - L(f,P) &= \sum_{j=1}^{n} (M_{j}-m_{j})\abs{I_{j}}.
    \end{align}
    Now, $f|_{I_{j}} : I_{j} \to \R$ is a continuous function for all valid $j$. Therefore, there exist $\eta_{j}, \zeta_{j} \in I_{j}$ such that $f(\eta_{j}) = M_{j}$ and $f(\zeta_{j}) = m_{j}$. $M_{j}-m_{j}$ can be rewritten as $f(\eta_{j}) - f(\zeta_{j})$. As $\abs{\eta_{j}-\zeta_{j}} < \delta$, it follows that
    \begin{align}
        M_{j}-m_{j} &< \frac{\varepsilon}{b-a} \text{ for all } j \\
        \implies (M_{j}-m_{j})\abs{I_{j}} &< \frac{\varepsilon}{b-a} \abs{I_{j}} \notag \\
        \implies \sum_{j=1}^{n} (M_{j}-m_{j}) \abs{I_{j}} &< \varepsilon.
    \end{align}
    By Darboux's theorem, $f$ is Riemann integrable.
\end{proof}
We now wish to compute $\int_{a}^{b} f$. Our first attempt at this is the following theorem.
\begin{theorem}
    Let $f \in \cB[a,b]$. Then $f \in \cR[a,b]$ if and only if there exists a sequence $\{P_{n}\}_{n \geq 1} \subseteq \cP[a,b]$ such that
    \begin{equation}
        \lim_{n \to \infty} U(f,P_{n}) - L(f,P_{n}) = 0.
    \end{equation}
    Moreover, in this case,
    \begin{equation}
        \int_{a}^{b} f = \lim_{n \to \infty} U(f,P_{n}) = \lim_{n \to \infty} L(f,P_{n}).
    \end{equation}
\end{theorem}
\begin{proof}
    Let us first assume that $f$ is Riemann integrable. Thus, for $\varepsilon = \frac{1}{n}$, there exists $P_{n} \in \cP[a,b]$ such that
    \begin{equation}
        0 \leq U(f,P) - L(f,P) < \frac{1}{n}
        \implies U(f,P) - L(f,P) \to 0.
    \end{equation}
    For the converse, let $\varepsilon > 0$. There exists $N \in \N$ such that
    \begin{equation}
        0 \leq U(f,P_{n}) - L(f,P_{n}) < \varepsilon \text{ for all } n \geq N.
    \end{equation}
    Pick $n$ to be $N$. Thus, $U(f,P_{N}) - L(f,P_{N}) < \varepsilon$ must imply that $f \in \cR[a,b]$.

    Let us now show the computation of the integral. We have
    \begin{align}
        0 \leq U(f,P_{n}) - \overline{\int_{a}^{b}} = U(f,P_{n}) - \underline{\int_{a}^{b}} f \leq U(f,P_{n}) - L(f,P_{n}) &\to 0 \implies \\
        \implies U(f,P_{n}) &\to \int_{a}^{b}f.
    \end{align}
    Similarly,
    \begin{align}
        0 \leq \underline{\int_{a}^{b}} f - L(f,P_{n}) = \int_{a}^{b} f - U(f,P_{n}) + U(f,P_{n}) - L(f,P_{n}) &\to 0 \\
        \implies L(f,P_{n}) &\to \int_{a}^{b} f.
    \end{align}
\end{proof}
\begin{remark}
    Let $f \in \cB[a,b]$, and let $\{P_{n}\}_{n \geq 1} \subseteq \cP[a,b]$. If $L(f,P_{n}) \to \lambda$ and if $U(f,P_{n}) \to \lambda$. We then must have $f \in \cR[a,b]$ and $\int_{a}^{b} = \lambda$. This is reminiscent of Newton's method of integration.
\end{remark}
\begin{example}
    Let us compute $\int_{0}^{1}f$ where $f(x) = x^{2}$ on $[0,1]$. For all $n \in \N$, consider the partitions $P_{n} : 0 = x_{0} < x_{1} = \frac{1}{n} < x_{2} = \frac{2}{n} < \ldots < x_{n} = \frac{n}{n} = 1$. Thus, $I_{j} = \left[ \frac{j-1}{n}, \frac{j}{n} \right] \text{ for all } j = 1, \ldots, n$. We then have $M_{j} = (\frac{j}{n})^{2}$ and $m_{j} = (\frac{j-1}{n})^{2}$. The sums can be computed as
    \begin{align}
        U(f,P_{n}) &= \sum_{j=1}^{n} \frac{1}{n} \cdot \frac{j^{2}}{n^{2}} = \frac{1}{n^{3}} \sum_{j=1}^{n} j^{2} = \frac{(n+1)(2n+1)}{6n^{2}} &\to \frac{1}{3}, \\
        L(f,P_{n}) &= \sum_{j=1}^{n} \frac{1}{n} \cdot \frac{(j-1)^{2}}{n^{2}} = \frac{1}{n^{3}} \sum_{j=1}^{n} (j-1)^{2} = \frac{(n-1)(2n-1)}{6n^{2}} &\to \frac{1}{3}.
    \end{align}
    Both sums converge to $\frac{1}{3}$; $f$ is Riemann integrable and $\int_{0}^{1} f = \frac{1}{3}$.
\end{example}
\begin{example}
    We show that $\cC[a,b]$ is a proper subset of $\cR[a,b]$. Let us consider the function $f:[a,b] \to \R$ defined by
    \begin{equation}
        f(x) = \begin{cases}
        1 &\text{ if } 0 \leq x < \frac{1}{2}, \\
        \frac{1}{2} &\text{ if } x = \frac{1}{2}, \\
        0 &\text{ if } \frac{1}{2} < x \leq 1.
        \end{cases}
    \end{equation}
    Fix $\varepsilon > 0$. Choose the partition $P_{\varepsilon} : 0 < \frac{1}{2}-\varepsilon < \frac{1}{2} + \varepsilon < 1$. We then have $m_{1} = 1 = M_{1}$, $m_{2} = 0 = m_{3}$, and $M_{2} = 1$, $M_{3} = 0$. Therefore, we have
    \begin{equation}
        L(f,P) = \frac{1}{2}-\varepsilon, \; U(f,P) = \frac{1}{2} + \varepsilon.
    \end{equation}
    Finally,
    \begin{equation}
        U(f,P) - L(f,P) = 2\varepsilon < 3\varepsilon.
    \end{equation}
    $f$ is Riemann integrable, but is not a continuous function.
\end{example}


\begin{appendices}

\titleformat{\chapter}[display]
  {\normalfont\Large\bfseries}
  {\chaptername\ \thechapter}{20pt}{\Huge}

\titlespacing*{\chapter}{0pt}{20pt}{40pt}

\chapter{Appendix}
Extra content goes here.

\printindex

\end{appendices}

\end{document}
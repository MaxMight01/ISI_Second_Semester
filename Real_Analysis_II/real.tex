
\documentclass[15pt,a4paper]{book}

\usepackage{amsmath, amsthm, amssymb} 
\usepackage{graphicx} % For including graphics
\usepackage{hyperref} % For clickable links
\usepackage{bookmark} % Better control over bookmarks
\usepackage{geometry} % Customize page layout
\usepackage{xcolor} % Colors for text and graphics
\usepackage{enumitem} % Customizable lists
\usepackage{fancyhdr} % Header and footer
\usepackage{titlesec} % Custom section/chapter titles
\usepackage[toc,page]{appendix} % For the appendix
\usepackage{longtable} % For tables spanning multiple pages
\usepackage{mathrsfs} % For script fonts in math mode
\usepackage{tocloft} % Custom table of contents
\usepackage{datetime2} % For dates
\usepackage{caption} % For better control over captions
\usepackage{float} % Fine control over figure/table placement
\usepackage{imakeidx} % For index

% Custom Theorem Styles
\newtheorem{theorem}{Theorem}[chapter]
\newtheorem{lemma}[theorem]{Lemma}
\newtheorem{proposition}[theorem]{Proposition}
\newtheorem{corollary}[theorem]{Corollary}
\theoremstyle{definition}
\newtheorem{definition}[theorem]{Definition}
\newtheorem{example}[theorem]{Example}
\newtheorem{remark}[theorem]{Remark}

\renewcommand{\cftchapfont}{\normalfont} % Remove bold for chapter names
\renewcommand{\cftchappagefont}{\normalfont} % Remove bold for chapter page numbers
\renewcommand{\qedsymbol}{$\blacksquare$}
\newcommand{\eax}[1]{\emph{#1}\index{#1}} % Macro for emphasis and index
\newcommand{\abs}[1]{\left| #1 \right|} % Absolute value
\newcommand{\N}{\mathbb{N}} % Natural Numbers
\newcommand{\R}{\mathbb{R}} % Real numbers
\newcommand{\Z}{\mathbb{Z}} % Integers
\newcommand{\cR}{\mathcal{R}}
\newcommand{\cP}{\mathcal{P}}
\newcommand{\cB}{\mathcal{B}}


% Custom Notation List Environment
\newlist{notationlist}{description}{1}
\setlist[notationlist]{font=\bfseries,labelsep=1em}

% Geometry Settings
\geometry{
    top=2.5cm,
    bottom=2.5cm,
    left=2.5cm,
    right=2.5cm,
}

% Hyperref Colors
\hypersetup{
    colorlinks=true,
    linkcolor=black,
    urlcolor=cyan,
    citecolor=red
}

\renewcommand{\chaptermark}[1]{\markboth{#1}{}}
\renewcommand{\sectionmark}[1]{\markright{#1}}

% Custom Headers
\pagestyle{fancy}
\fancyhf{}
\fancyhead[L]{\leftmark} % Chapter name on top left
\fancyhead[R]{\rightmark}  % section name on top right
\fancyfoot[C]{\thepage}

\renewcommand{\headrulewidth}{0pt}
\renewcommand{\footrulewidth}{0pt}

% Making index
\makeindex[intoc]

% Title Formatting
\titleformat{\chapter}[display]
  {\normalfont\Large\bfseries \centering}
  {\chaptername\ \thechapter}{20pt}{\Huge \centering}

\titlespacing*{\chapter}{0pt}{20pt}{100pt}

\begin{document}

\pagestyle{empty}

\begin{titlepage}
    \begin{center}
    \vspace*{\fill}
    % Title in all caps
    {\Huge \textbf{\MakeUppercase{Real Analysis II}}\par}

    \vspace{0.5cm} % Adjust vertical spacing between title and subtitle
    % Subtitle in normal text, slightly enlarged
    {\Large Jaydeb Sarkar, notes by Ramdas Singh\par}

    \vspace{0.5cm} % Additional spacing before the author
    % Author information
    {\large Second Semester\par}
    \vspace*{\fill}
    \end{center}
\end{titlepage}

\clearpage

\pagenumbering{roman}

\chapter*{List of Symbols}
\begin{notationlist}
    \item $[a,b]$, the set of all real numbers $x$ such that $a \leq x \leq b$.
    \item $\mathbb{N} = \{1,2,\ldots\}$, the set of all natural numbers.
    \item $\mathbb{Z}_{+}$, defined as $\mathbb{N} \cup \{0\}$.
    \item $\mathcal{B}[a,b]$, the set of all boundary functions defined as $\{f:[a,b] \to \mathbb{R}\}$. It is a vector space (also an algebra) over $\mathbb{R}$.
    \item $\mathcal{P}[a,b]$, the set of all partitions of the set $[a,b]$.
    \item $I_{j}$, the $j^{\text{th}}$ subinterval of $[a,b]$, controlled by a partition set.
    \item $L(f,P)$, the lower Riemann sum for a function $f$ and partition $P$.
    \item $U(f,P)$, the upper Riemann sum for a function $f$ and partition $P$.
    \item $\int_{\underline{a}}^{b} f$, the lower Riemann integration for a function $f$.
    \item $\int_{a}^{\overline{b}} f$, the upper Riemann integration for a function $f$.
    \item $\mathcal{R}[a,b]$, the set of all Riemann integrable functions over the set $[a,b]$.
\end{notationlist}

\newpage
\setcounter{tocdepth}{2}
\tableofcontents

\newpage
\pagenumbering{arabic}
\pagestyle{fancy}


%%-------------------------------------------------------------------------------------------------

\chapter{THE RIEMANN INTEGRAL}

\section{On The Path of Definitions}

\textit{January 6th.}

\begin{definition}
    A \eax{partition} of $[a,b]$ are all the points $a=x_{0}<x_{1}<\ldots<x_{n} = b$. These points within are termed \eax{nodes}, and there are $n-1$ of them. The set $I_{j}$, defined by $[x_{j-1},x_{j}]$ denotes the $j^{\text{th}}$ subinterval.
\end{definition}

\begin{definition}
    If $I = (a,b), [a,b], (a,b], [b,a)$, then the \eax{length of the interval} $I$ is denoted by $b-a$.
\end{definition}

Denote by $\mathcal{P}[a,b]$, the set of all partition sets of $[a,b]$. For $P \in \mathcal{P}[a,b]$, with $n-1$ nodes, the length of $[a,b]$ will be $\abs{[a,b]} = \sum_{j=1}^{n} I_{j}$. We also note that for all $P, \tilde{P} \in \mathcal{P}[a,b]$, $P \cup \tilde{P} \in \mathcal{P}[a,b]$. Note that here we consider $n$ to be finite.

\begin{example}
    The set $\{\frac{1}{n}\}_{n \geq 1} \cup \{0\}$ does not belong to the set of all partitions of the unit interval, $\mathcal{P}[0,1]$.
\end{example}

Let $f \in \mathcal{B}[a,b]$, and $P \in \mathcal{P}[a,b]$. Suppose $P$ has the nodes $a = x_{0} < x_{1} < \ldots < x_{n} = b$. For all $j = 1, \ldots n$, define $m_{j} = \inf_{x \in I_{j}} f(x)$ and $M_{j} = \sup_{x \in I_{j}} f(x)$. Finally, denote by $m$ the value of $\inf_{x \in [a,b]} f(x)$ and $M$ to be $\sup_{x \in [a,b]} f(x)$. These are all real values.

Note that for all valid $j$, $m \leq m_{j} \leq M_{j} \leq M$ always holds. This must mean that
\begin{align}
    m \abs{I_{j}} &\leq m_{j} \abs{I_{j}} \leq M_{j} \abs{I_{j}} \leq M \abs{I_{j}} \notag \\
    m(b-a) &\leq \sum_{j=1}^{n} m_{j} \abs{I_{j}} \leq \sum_{j=1}^{n} M_{j} \abs{I_{j}} \leq M(b-a).
\end{align}

\begin{definition}
    Let $f \in \mathcal{B}[a,b]$. For $P$ $(a = x_{0}, x_{1}, \ldots, x_{n} = b)$ $\in \mathcal{P}[a,b]$, the \eax{lower Riemann sum} and the \eax{upper Riemann sum} are defined as
    \begin{equation}
        L(f,P) = \sum_{j=1}^{n} m_{j} \abs{I_{j}} \text{ and } U(f,P) = \sum_{j=1}^{n} M_{j} \abs{I_{j}},
    \end{equation}
    respectively. Thus, $m(b-a) \leq L(f,P) \leq U(f,P) \leq M(b-a)$ $\forall$ $P \in \mathcal{P}[a,b]$.
\end{definition}

\begin{remark}
    Clearly, $L(f,P), U(f,P) \in \mathbb{R}$ for all paritions $P \in \mathcal{P}[a,b]$ and all boundary functions $f \in \mathcal{B}[a,b]$. In fact, $L(f,P), U(f,P) \in [m(b-a),M(b-a)]$.
\end{remark}

\begin{definition}
    For $f \in \mathcal{B}[a,b]$, the \eax{lower Riemann integration} is defined as
    \begin{equation}
        \int_{\underline{a}}^{b} f = \sup\{L(f,P) | P \in \mathcal{P}[a,b]\}.
    \end{equation}
    Subsequently, the \eax{upper Riemann integration} is defined as
    \begin{equation}
        \int_{a}^{\overline{b}} f = \inf\{U(f,P) | P \in \mathcal{P}[a,b]\}.
    \end{equation}
\end{definition}
\begin{remark}
    Note that both $\int_{\underline{a}}^{b} f$ and $\int_{a}^{\overline{b}} f$ belong to the set $[m(b-a),M(b-a)]$.
\end{remark}

\begin{definition}
    A function $f \in \mathcal{B}[a,b]$ is \eax{Riemann integrable} if the lower and the upper Riemann integration are equal, that is, $\int_{\underline{a}}^{b} f = \int_{a}^{\overline{b}} f$. We denote this value by $\int_{a}^{b} f$, and call it the integration of $f$ over $[a,b]$. We then say that $f \in \mathcal{R}[a,b]$.
\end{definition}





\begin{appendices}

\titleformat{\chapter}[display]
  {\normalfont\Large\bfseries}
  {\chaptername\ \thechapter}{20pt}{\Huge}

\titlespacing*{\chapter}{0pt}{20pt}{40pt}

\chapter{Appendix}
Extra content goes here.

\printindex

\end{appendices}

\end{document}